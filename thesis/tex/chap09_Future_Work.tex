%!TEX root = thesis.tex

\chapter{Future Work}
\label{chp:Future_Work}

A central and urgent area for future work arises directly from the on-device adaptation mechanism of \gls{tinylcm}, which was conceptually designed in Chapter~\ref{chp:Framework} (Section~\ref{ssec:tinylcm_drift_adaptation}) and highlighted as a limitation in Chapter~\ref{chp:Discussion}. The ability of a \gls{tinyml} system to autonomously react to detected concept drifts and adapt its model is crucial for maintaining long-term performance and reliability. Future efforts in this domain should concentrate on several key aspects. Firstly, the development and implementation of efficient algorithms for pseudo-label generation is paramount. The heuristic rules and clustering approaches envisioned for the \texttt{HeuristicAdapter} to generate pseudo-labels for anomalous samples collected in the \texttt{QuarantineBuffer} must be meticulously elaborated and implemented in a resource-efficient manner. This necessitates research into lightweight clustering methods suitable for execution on \glspl{mcu} or \glspl{sbc} with limited memory and computational power. Secondly, a robust update of the on-device classifier is required. The mechanisms within the \texttt{AdaptiveHandler} for cautiously updating the \texttt{LightweightKNN} classifier with pseudo-labeled data must be refined, with strategies to prevent catastrophic forgetting and to control the integration of new information being particularly important. Lastly, rigorous empirical validation of the implemented adaptation mechanisms must be undertaken. This includes testing the effectiveness of adaptation to various types of concept drift (abrupt, gradual, recurrent), investigating the long-term stability of the learning process, and analyzing the impact on overall system performance.

In parallel with advancing \gls{tinylcm}'s on-device capabilities, the further development and evaluation of the TinySphere platform and its associated feedback loop are of considerable importance. Future work concerning TinySphere should encompass several directions. A primary focus will be the refinement of interfaces and data exchange mechanisms; the communication protocols and data formats for exchange between \gls{tinylcm} and TinySphere must be optimized to ensure efficiency, robustness, and security, particularly considering intermittent connectivity. Furthermore, the implementation of advanced analysis methods within TinySphere is necessary; it should be equipped with analytical tools to evaluate aggregated data from device fleets, potentially including the automatic detection of global drift patterns, comparative analysis of model performance across different devices and environments, or the identification of systematic error sources. Another key area is the robust implementation of the human-in-the-loop validation workflow. The process for validating on-device generated pseudo-labels by human experts, as outlined in Chapter~\ref{chp:Framework} (Section~\ref{ssec:tinysphere_capabilities}), must be fully implemented. Additionally, the drift classification could also be achived fully automated, by integration better models on server-side, combined with automated orchestration tools such as Airflow.

The empirical evaluation conducted in Chapter~\ref{chp:Evaluation} provided important initial results but is limited in scope. Thus, a significant expansion of this empirical evaluation is required to more comprehensively investigate the robustness and generalizability of the developed solutions. One critical aspect of this expansion is the investigation on more resource-constrained \glspl{mcu}. The evaluation of \gls{tinylcm} to date has been performed on a Raspberry Pi Zero 2W, a relatively capable \gls{sbc}. To demonstrate the framework's applicability to the lower spectrum of \gls{tinyml} hardware, porting and evaluating \gls{tinylcm}, or at least its core components, on typical \glspl{mcu} is essential. This will likely require reimplementation in C/C++, introducing new challenges regarding memory management, computation time optimization, and integration with bare-metal or \gls{rtos} environments. Such investigations would yield valuable insights into the scalability and actual minimum resource requirements of the approach. Furthermore, the applicability to more complex datasets and tasks, such as other sensor data (e.g., time series from accelerometers, audio data) and more demanding \gls{ml} assignments, should be explored. This requires the development of new, more challenging test scenarios and potentially the use of established benchmarking datasets for concept drift detection.

Beyond the core functionalities already addressed, an investigation of additional \gls{mlops} aspects would enhance the practical viability of the \gls{tinymlops} ecosystem. Security and data privacy in on-device operations and data synchronization are critical cross-cutting themes for trustworthy \gls{ai} systems. Future research in this area should address several facets. These include the development of robust mechanisms for secure model and software updates via \gls{ota}, covering \gls{ml} models, adaptation logic, and the \gls{tinylcm} firmware itself. Additionally, an investigation of the vulnerability of on-device models to malicious adversarial attacks and the development of lightweight defense strategies are needed. Finally, ensuring privacy-compliant data aggregation and analysis is essential; this means that the synchronization of data with TinySphere and its analysis must comply with data privacy requirements, for example, through anonymization or pseudonymization techniques, or the use of privacy-preserving machine learning methods. This beed done, \gls{fl} could be explored by enabling collaborative learning across devices without the need for constant data transmission to a central server. This would not only enhance energy efficiency but also improve data privacy by keeping sensitive data on the device. Another vital consideration is the energy efficiency of on-device learning and adaptation processes. Many \gls{tinyml} devices are battery-operated, making energy efficiency a dominant design factor. Future work must meticulously analyze and optimize the energy consumption of on-device learning and adaptation processes, as well as continuous drift monitoring. This could involve the development of energy-aware algorithms, adaptive sampling rates for monitoring, or the utilization of hardware accelerators.

Finally, although the Mars rover mission served as a motivating scenario, future work should investigate the potential application fields and transferability of the results of the developed \gls{tinymlops} ecosystem to other domains. Potential application fields include, but are not limited to, industrial condition monitoring and predictive maintenance, smart agriculture for optimized resource management, personalized medical technology and health monitoring, and autonomous systems in logistics or environmental monitoring. Investigating the specific requirements of these application fields and adapting the ecosystem would demonstrate its broader relevance and the impact potential of the research.

In summary, the present work has laid a solid foundation. The outlined future work aims to bridge the conceptual-empirical gap, enhance the generalizability and robustness of the solution, and integrate important cross-cutting themes such as security and energy efficiency to further increase the practical viability of the \gls{tinymlops} ecosystem.
