%%%
%%% Own commands and environments
%%%

% page clearing
\newcommand{\clearemptydoublepage}{%
  \ifthenelse{\boolean{@twoside}}{\newpage{\pagestyle{empty}\cleardoublepage}}%
  {\clearpage}}

%% color definition
%  red, green, blue, white, and black come from color.sty
\definecolor{brightgray}{gray}{0.9}
\definecolor{darkgray}{gray}{0.5}
\definecolor{darkgreen}{rgb}{0.0,0.5,0.0}
\definecolor{dg}{rgb}{0.0,0.5,0.0}  %% dark green (comments in listings)
\definecolor{dr}{rgb}{0.5,0.0,0.0}  %% dark red

%% Re-formatting paragraph spacing
%  Paragrpahs have 0pt indent, but are now
%  vertically separated
\setlength\parskip{\smallskipamount}
\setlength\parindent{0pt}

%% Re-definition Typewriter Font = smaller font size (more harmonic text) and hyphenation support
\renewcommand\texttt[1]{\small\ttfamily\hyphenchar\font=\defaulthyphenchar #1{}\normalsize\rmfamily\hyphenchar\font=\defaulthyphenchar}

%% definition of standard listing font
\lstset{language=csharp,extendedchars=true,basicstyle=\footnotesize\ttfamily,commentstyle=\itshape,breaklines=true}

%% Easy switch between different programming languages
%  and the font, which is used to type set the listing
%  Parameter:
%  1: Name of the language, e.g., csharp
%  2: Name of the font family, e.g., \ttfamity
%  3: Name of the font family for comments, e.g., \itshape
%
%  Beispiel: \setlistingstyle{csharp}{\ttfamily}{\itshape}
\newcommand{\setlistingstyle}[3]{%
\lstset{language=#1,extendedchars=true,basicstyle=\footnotesize#2,commentstyle=#3,breaklines=true}
}

%% Redet type setting of listig to the standard values
%  
\newcommand{\resetlistingstyle}{%
\lstset{language=csharp,extendedchars=true,basicstyle=\footnotesize\ttfamily,commentstyle=\itshape,breaklines=true}
}

%% Re-definition: Formats and fonts for caption labels
\addtokomafont{caption}{\sffamily\small}
\setkomafont{captionlabel}{\sffamily\bfseries}
\setkomafont{descriptionlabel}{\sffamily\bfseries\small}

%% Outline \part is type set without prefix, i.e. "Part"
\renewcommand*{\partformat}{\thepart\autodot}
\deffootnote{1em}{1em}{\thefootnotemark\ }

%% New theorems
\newtheoremstyle{style}
   {}                   %Space above
   {}                   %Space below
   {}                   %Body font: original {\normalfont}
   {}                   %Indent amount (empty = no indent,
                           %\parindent = para indent)
   {\normalfont\sffamily}  %Thm head font original {\normalfont\bfseries}
   {:}                     %Punctuation after thm head original :
   {\newline}              %Space after thm head: " " = normal interword
                           %space; \newline = linebreak
   {\textbf{\thmname{#1}\thmnumber{ #2}\thmnote{ (#3)}}}                    
                                        %Thm head spec (can be left empty, meaning
                           %`normal') original {\underline{\thmname{#1}\thmnumber{ #2}\thmnote{ (#3)}}}
 
\theoremstyle{style} 
\newtheorem{defi}{Definition}[chapter]
\newtheorem{bsp}{Example}[chapter]

%% type set font: small + sans serif
\newcommand{\textsmf}[1]{\small{\textsf{#1}}}
%\newcommand{\hl}[1]{\textsf{#1}}

\newcommand{\hl}[1]{\small\sffamily\bfseries\hyphenchar\font=\defaulthyphenchar #1{}\normalsize\rmfamily\mdseries\hyphenchar\font=\defaulthyphenchar}

%% type set font: emphasized + sans serif
\newcommand{\textei}[1]{\emph{\textsf{#1}}}

%% New environments:
%  MySample: Environment for inline examples that are highlighted:
%            different font face and adjusted indents
\newenvironment{MySample}{%
  \begin{labeling}[:]{%
      \usekomafont{descriptionlabel}Sample}
  \item[\usekomafont{descriptionlabel}Sample]\usekomafont{caption}
    }{%
    \normalfont \end{labeling}
}

%  MySugg: Environment for inline hints/suggestions that are highlighted:
%          different font face and adjusted indents
\newenvironment{MySugg}{%
  \begin{labeling}[:]{%
      \usekomafont{descriptionlabel}Hint}
  \item[\usekomafont{descriptionlabel}Hint]\usekomafont{caption}
    }{%
    \normalfont \end{labeling}
}


%% MyBox: Environment for emphasized text. Different font face,
%         adjusted coloring, and adjusted indents.
\newenvironment{MyBox}[1]{%
    \color{darkgray}%
    \rule{\linewidth}{1pt}%
    \color{black}%
    \usekomafont{descriptionlabel}
    \textcolor{darkgray}{{#1}}

    \begin{addmargin}[2em]{0pt}%
    \usekomafont{caption}%
}{%
  \end{addmargin}%
  \color{darkgray}%
  \rule[.5\baselineskip]{\linewidth}{1pt}%
  \normalfont%
}

%% verbatin in footnode
\VerbatimFootnotes